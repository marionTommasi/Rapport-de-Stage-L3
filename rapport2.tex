\documentclass[a4paper,12pt,final]{article}
% Pour une impression recto verso, utilisez plutôt ce documentclass :
%\documentclass[a4paper,11pt,twoside,final]{article}

\usepackage[english,francais]{babel}
\usepackage[utf8]{inputenc}
\usepackage[T1]{fontenc}
\usepackage[pdftex]{graphicx}
\usepackage{setspace}
\usepackage{xcolor}
\usepackage[hidelinks]{hyperref}
\usepackage[french]{varioref}
\usepackage{algorithm}
\usepackage{algorithmic}
%\usepackage[ruled,vlined]{algorithm2e}

\newcommand{\reporttitle}{Algorithme de recherche dans des données de séquençage ADN avec la transformée de Burrows-Wheeler à contexte borné}     % Titre
\newcommand{\reportauthor}{Marion \textsc{Tommasi}} % Auteur
\newcommand{\reportsubject}{Stage de Licence 3 informatique} % Sujet
\newcommand{\HRule}{\rule{\linewidth}{0.5mm}}
\setlength{\parskip}{1ex} % Espace entre les paragraphes
%\renewcommand{\baselinestretch}{1.5}

\newcommand{\adn}{\textsc{adn\xspace}}
\newcommand{\bwt}{\textsc{bwt\xspace}}
\newcommand{\kbwt}{$k$-\textsc{bwt\xspace}}

\hypersetup{
    pdftitle={\reporttitle},%
    pdfauthor={\reportauthor},%
    pdfsubject={\reportsubject},%
    pdfkeywords={rapport} {L3} {bio-informatique} {algorithmique du texte}
}

\begin{document}
  % Inspiré de http://en.wikibooks.org/wiki/LaTeX/Title_Creation

\begin{titlepage}

\begin{center}

\begin{minipage}[t]{0.48\textwidth}
  \begin{flushleft}
%    \includegraphics [width=30mm]{images/logo-univ.png} \\[0.5cm]
    \begin{spacing}{1.5}
      \textsc{\large Université de Lille 1}
    \end{spacing}
  \end{flushleft}
\end{minipage}
%{\includegraphics [width=30mm]{images/INRIA_logo.png} \\[0.5cm]}
\begin{minipage}[t]{0.48\textwidth}
  \begin{flushright}
%    \includegraphics [width=30mm]{images/INRIA_logo.png} \\[0.5cm]
    \textsc{\large Équipe Bonsai}
  \end{flushright}
\end{minipage} \\[1.5cm]

%\begin{minipage}[t]{0.48\textwidth}
%    \includegraphics [width=30mm]{images/logo-univ.png} \\[0.5cm]
%    \includegraphics [width=30mm]{images/INRIA_logo.png} \\[0.5cm]
%\end{minipage}
%
%\begin{minipage}[t]{0.48\textwidth}
%  \begin{flushright}
%	\textsc{\large Université de Lille 1}
%  \end{flushright}
%  \begin{flushleft}
%    \textsc{\large Équipe Bonsai}
%  \end{flushleft}
%\end{minipage}


\textsc{\Large \reportsubject}\\[0.5cm]
\HRule \\[0.4cm]
{\LARGE \bfseries \reporttitle}\\[0.4cm]
\HRule \\[1.5cm]

\begin{minipage}[t]{0.3\textwidth}
  \begin{flushleft} \large
    \emph{Auteur :}\\
    \reportauthor
  \end{flushleft}
\end{minipage}
\begin{minipage}[t]{0.6\textwidth}
  \begin{flushright} \large
    \emph{Responsables :} \\
    Rayan \textsc{Chikhi} \\
    Mikael \textsc{Salson}
  \end{flushright}
\end{minipage}

\vfill

{\large 30 mars - 30 juin 2015}

\end{center}

\end{titlepage}

  \cleardoublepage % Dans le cas du recto verso, ajoute une page blanche si besoin
  \tableofcontents % Table des matières
  \sloppy          % Justification moins stricte : des mots ne dépasseront pas des paragraphes
  \cleardoublepage
  \section*{Remerciements}
\addcontentsline{toc}{section}{Remerciements}

Je tiens à remercier toutes les personnes qui m'ont aidées pendant ce stage et à la rédaction de ce rapport.

Tout d'abord, je voudrais remercier mes maîtres de stage Rayan Chikhi et Mikael Salson pour m'avoir aider et guider tout au long de ce stage, ainsi que toute l'équipe Bonsai pour m'avoir chaleureusement accueillie en son sein.

Enfin, je tiens à remercier \LaTeX{} et les tutoriels sur internet\,\footnotemark{} pour l'aide apportée à la construction de ce rapport.


\footnotetext{
http://www.ukonline.be/programmation/latex/tutoriel/index.php et \\
http://blog.hikoweb.net/index.php?/post2011/11/06/Exemple-de-rapport-en-LaTeX}


%\uparrow\cite{tutoltx}\cite{expltx}}

%Voici une note\,\footnote{Texte de bas de page} de bas de page.
%Une deuxième\,\footnotemark{} déclarée différemment.
%La même note\,\footnotemark[\value{footnote}].
%
%\footnotetext{Il a deux références vers cette note}
  \cleardoublepage
  \section*{Résumé} % Pas de numérotation
\addcontentsline{toc}{section}{Résumé} % Ajout dans la table des matières

Le séquençage de l'ADN est depuis plus de 20 ans un sujet important de la bio-informatique, avec des applications en biologie fondamentale comme en médecine. Aussi bien pour étudier l'évolution ou des maladies, la demande est de plus en plus importante. De plus, depuis l'émergence des techniques de séquençage haut débit, la masse de données biologiques a augmenté plus vite que l'efficacité des algorithmes existants. Le besoin d'améliorer ces algorithmes est donc de plus en plus pressant, à chaque étape du traitement et de l'analyse de ces données. La compression et l'indexation des reads, fragments d'ADN retournés par les séquenceurs, est notamment un sujet pour lequel le besoin de nouvelles méthodes est important, du fait de la taille grandissante de ces données.

Beaucoup d'algorithmes sur la compression de données existent déjà, et sont utilisés pour stocker des données biologiques. Le travail présenté ici porte donc sur l'étude de l'un de ces algorithmes, \textit{la transformée de Burrows-Wheeler à contexte borné}, et son amélioration pour ce type de données particulier que sont les reads. Ce travail a été réalisé au cours du stage de fin de licence informatique à Lille 1, dans l'équipe de bio-informatique Bonsai.

%Dans le cadre de ma troisième année de licence informatique, j'ai effectué un stage de recherche à l'Inria. Du 30 mars au 30 juin, dans l'équipe Bonsai, j'ai travaillé sur l'algorithmique du texte appliquée à la bio-informatique.

%Ce stage a été pour moi l'opportunité d'approfondir mes connaissances sur la bio-informatique, ainsi que sur diverses structures de données telles que les arbres ou les tableaux de suffixes. Il m'a également permis de découvrir le monde de la recherche et de participer à des conférences.

%Mon travail a consisté essentiellement en l'étude, l'amélioration et l'implémentation d'un algorithme de compression et d'indexation de texte.


%Afin de rendre de compte de ce stage, il parait important de présenter l'équipe dans lequel il s'est déroulé, ainsi que le champ de recherche sur lequel il portait. Il apparaît ensuite nécessaire d'expliquer l'état de l'art concernant le problème posé par le sujet de stage. On pourra alors exposer le travail réaliser pendant ce stage, en revenant d'abord sur le sujet, puis en expliquant les pistes suivies, pour enfin s'intéresser aux résultats.
  \cleardoublepage
  \section{Contexte}

%
%\subsection{L'équipe}
%
%Mon stage s'est déroulé au sein de l'équipe Bonsai, équipe de recherche affiliée à l'Inria Lille - Nord Europe et le Centre de Recherche en Informatique, Signal et Automatique de Lille (CRIStAL, Université Lille 1, CNRS). 
%
%\subsubsection{Bonsai}
%L'équipe Bonsai compte 20 personnes, dont 10 membres permanents, sous la responsabilité d'Hélène Touzet. Le principal objectif de l'équipe est de développer des algorithmes d'analyse de séquences biologiques, que ce soit pour l'étude de l'ADN, de l'ARN, des protéines ou des NRP (peptides non-ribosomiques). Ce travail se réalise souvent en collaborations avec des biologistes, et se concrétise par la création de logiciels. Vidgil, par exemple, est une plate-forme  d'analyse des données de séquençage haut débit développée par l'équipe et utilisée à l'institut Pasteur pour suivre l'évolution de leucémies.
%


\subsection{Séquençage}
Le séquençage consiste à déterminer l'ordre des composants de molécules. Le séquençage de l'ADN sert à déterminer l'ordre des nucléotides (Guanine, Thymine, Cytosine ou Adénine) composant un fragment donné. Pour cela, diverses techniques existent. Historiquement, la méthode la plus utilisée a été celle dite de Sanger, d'après son inventeur, plus facile à robotiser que celle de Maxam et Gilbert, développée à la même époque. Cependant, depuis quelques années, de nouvelles méthodes plus efficaces de séquençage haut débit sont apparues, multipliant le nombre de données issues du séquençage.

\subsubsection{Méthode de Sanger}
L'ADN polymérase est une enzyme utilisée dans la réplication de l'ADN, qui permet de créer des molécules d'ADN en assemblant des nucléotides. Les nucléotides utilisés lors de la réplication sont des désoxyribonucléotides (dNTP), mais il existe aussi des didésoxyribonucléotides (ddNTP), qui forcent l’arrêt de la réaction à la suite de leur utilisation. 

La méthode de Sanger consiste à répliquer un brin d'ADN dans un milieu où se trouvent des dNTP ainsi que de quelques ddNTP d'un seul type (ddNTP à Guanine, Thymine, Adénine ou Cytosine). La réplication va donc s’arrêter aléatoirement sur l'un des nucléotides correspondant à la ddNTP choisie. Cette expérience est répétée quatre fois, avec les quatre types de ddNTP, puis l'on compare les longueurs des brins d'ADN répliqués par chromatographie. Il faut ensuite lire la séquence ADN, processus automatisable mais long.


\begin{figure}[!ht]
    \center
    \includegraphics[]{./images/chromato_acanthoweb.jpg}
    \caption{Exemple de chromatographie | Source : ancanthoweb.fr}
    \label{chromato}
\end{figure}


\subsubsection{Séquençage haut débit}
Il existe de nombreuses techniques de séquençage haut débit, plus ou moins chères, fiables ou efficaces.
Par exemple, le séquençage par synthèse, utilisé par l'entreprise \emph{Illumina}, est basé sur l'utilisation de colorants fluorescents.

Pour cette méthode, on attache aux séquences étudiées des amorces, qui servent à la synthèse de l'ADN. Puis avec l'ajout de nucléotides et d'ADN polymérase, ces séquences sont amplifiées. On ajoute enfin des nucléotides associés à des couleurs spécifiant leur type (Guanine, Thymine, Cytosine et Adénine), qui vont s'associer à leur complémentaire sur les brins d'ADN. Les nucléotides non-attachés sont ensuite enlevés, et les colorants excités et photographiés. Les colorants sont retirés chimiquement et le processus répété plusieurs fois jusqu'à ce que l'ensemble de l'ADN soit séquencé.

\subsubsection{Reads}
Les données retournées par les séquenceurs sont appelées des \emph{reads}. Ce sont de très nombreux fragments de l'ADN séquencé. La taille et le nombre de ces fragments varient, mais ils sont souvent composés de 100 à 300 nucléotides et couvrent au moins 30 à 40 fois la séquence initiale. 

Ce sont donc des données très redondantes, ce que nous chercherons à exploiter dans ce stage.


\subsection{Structures de données utilisées}
La suite de ce rapport se concentrera sur une façon de compresser et indexer les reads. Pour cela, nous allons utiliser plusieurs structures de données, notamment des tableaux de suffixes et des wavelet tree que nous présentons ici.

\subsubsection{Tableau de suffixes}
Un suffixe d'une chaîne de caractères $s$ est une chaîne $s'$ telle qu'il existe une chaîne $p$, pouvant être vide, satisfaisant $p + s' = s$ ($+$ représentant la concaténation des chaînes). $p$ est d'ailleurs un préfixe de $s$.
%Un suffixe d'une chaîne de caractères $s$ est une chaîne $s'$ plus petite telle que $s'$ termine $s$. C'est-à-dire tel qu'il existe une chaîne $p$, pouvant être vide, telle que $p + s' = s$, ($+$ représentant la concaténation des chaînes). $p$ est d'ailleurs un préfixe de $s$.

Un tableau de suffixes est un tableau d'entiers représentant toutes les positions des suffixes d'une chaîne. Pour le construire, il faut construire l'ensemble des suffixes, en attribuant à chaque suffixe un entier présentant sa position dans le texte. Les suffixes sont ensuite triés par ordre lexicographique, et seule leur position est gardée. La figure \ref{suffixes} illustre cette construction.

\begin{figure}[h!]
\fbox{%
  \begin{minipage}{0.4\linewidth}
  	\begin{flushleft}
      \textsc{%
  	  \begin{tabular}{l|c|}
  	  	suffixe & position \tabularnewline \hline
        barbapapa\$ & 0\\
	    arbapapa\$ & 1\\
	    rbapapa\$ & 2\\
	    bapapa\$ & 3\\
	    apapa\$ & 4\\
	    papa\$ & 5\\
	    apa\$ & 6\\
	    pa\$ & 7\\
        a\$ & 8\\
	    \$ & 9
	  \end{tabular} 
	  }
  	\end{flushleft}
  \end{minipage}%
  \begin{minipage}{0.2\linewidth}
	\begin{center}
		tri\\
  		$\rightarrow$
	\end{center}
  \end{minipage}%
  \begin{minipage}{0.4\linewidth}
  	\begin{flushright}
      \textsc{%
  	  \begin{tabular}{|l|c}
  	  	suffixe & position \tabularnewline \hline
	    \$ & {\color{red}9}\\
        a\$ & {\color{red}8}\\
	    apa\$ & {\color{red}6}\\
	    apapa\$ & {\color{red}4}\\
	    arbapapa\$ & {\color{red}1}\\
	    bapapa\$ & {\color{red}3}\\
	    barbapapa\$ & {\color{red}0}\\
	    pa\$ & {\color{red}7}\\
        papa\$ & {\color{red}5}\\
	    rbapapa\$ & {\color{red}2}\\
	  \end{tabular} 
	  }
  	\end{flushright}
  \end{minipage}%
}
\caption{Construction du tableau des suffixes (en rouge) pour la chaîne "\textsc{barbapapa\$}"}
\label{suffixes}
\end{figure}

\subsubsection{Wavelet tree}
Les arbres sont des structures de données couramment utilisées en informatique, représentant des graphes acycliques orientés, tels que tous les nœuds ont un unique parent, sauf la racine qui n'a pas de parent. Pour les arbres binaires, comme le wavelet tree, les nœuds ont au plus deux fils. Les nœuds n'ayant pas de fils sont appelés feuilles.

Le \textit{wavelet tree}\up{\cite{wt}}, ou arbre d'ondelettes, est une structure de données succincte (c'est-à-dire stockée optimalement), à l'origine utilisée pour représenter les tableaux de suffixes compressés.

La construction du wavelet tree est basée sur la division de l'alphabet du mot d'origine en deux. Cette division peut se faire par ordre lexicographique, ou par d'autres algorithmes plus complexes tels que le codage de Huffman. Nous ne nous intéresserons ici qu'à la première façon.

Prenons comme exemple la construction du wavelet tree de \textsc{"barbapapa\$"} (Figure \ref{wavelet}). L'alphabet est l'ensemble \textsc{\{'\$'; 'a'; 'b'; 'p'; 'r'\}}. Les lettres appartenant à la première moitié de l'alphabet (\textsc{\{'\$'; 'a'\}}) seront représentées par un 0, tandis que les lettres de la deuxième moitié seront des 1. La racine comporte donc le mot "1011010100".

Les deux moitiés de l'alphabet descendent dans les fils de la racine. Le premier nœud code donc \textsc{"aaaa\$"} sur l'alphabet \textsc{\{'\$'; 'a'\}}, et le deuxième nœud code \textsc{"brbpp"} sur l'alphabet \textsc{\{'b'; 'r'; 'p'\}}. La même opération que sur la racine est alors renouvelée, jusqu'à ce que les nœuds ne codent plus qu'une seule lettre.

\begin{figure}[h]
\Tree [.{\begin{tabular}{|c c c c c c c c c c|}
\hline
\textsc{b}&\textsc{a}&\textsc{r}&\textsc{b}&\textsc{a}&
\textsc{p}&\textsc{a}&\textsc{p}&\textsc{a}&\textsc{\$} \\
1&0&1&1&0&1&0&1&0&0\\
\hline
\end{tabular}}
 		[.{\begin{tabular}{|c c c c c|}
 		   \hline
		   \textsc{a}&\textsc{a}&\textsc{a}&\textsc{a}&\$ \\
		   1&1&1&1&0 \\
		   \hline
		   \end{tabular}}
 		  \textsc{\$}
 		  \textsc{aaaa}
 		]  
 		[.{\begin{tabular}{|c c c c c|}
 			\hline
			\textsc{b}&\textsc{r}&\textsc{b}&\textsc{p}&\textsc{p} \\
			0&1&0&1&1 \\
		    \hline
			\end{tabular}} 
 			\textsc{bb}
 			[.{\begin{tabular}{|c c c|}
 				\hline
	   			\textsc{r}&\textsc{p}&\textsc{p} \\
	   			1&0&0 \\
		   		\hline
	   			\end{tabular}} 
	   			\textsc{pp}
	   			\textsc{r}
			]
 		]
 	]

\caption{wavelet tree de \textsc{barbapapa\$}}
\label{wavelet}
\end{figure}


\subsection{Transformée de Burrows-Wheeler}

La transformée de Burrows-Wheeler\up{\cite{bwt}}, abrégée \bwt, est un algorithme de réorganisation des données, fréquemment utilisé pour compresser les données. C'est un algorithme réversible qui augmente la probabilité que des caractères identiques se retrouvent côte à côte. Il est notamment utilisé dans l'algorithme de compression \emph{bzip2}.

\subsubsection{Construction}
Une rotation de $k$ éléments d'une chaîne de caractères de longueur $n$ est composée des $n-k$ derniers éléments de la chaîne concaténés aux $k$ premiers. Par exemple, pour une rotation de 3 sur la chaîne \textsc{"barbapapa"}, on part de la troisième lettre, le \textsc{"r"} et on lit toutes les lettres jusqu'à retomber sur le \textsc{"r"} de départ, en liant la fin du mot avec son début : \textsc{"rbapapa~ba"}.

Pour construire la \bwt, on prend la dernière lettre de toutes les rotations de la chaîne initiale triées lexicographiquement.
Pour assurer la réversibilité de la \bwt, on ajoute un caractère spécial à la fin de la chaîne, lexicographiquement plus petit que tous les autres et traditionnellement représenté par \$. Grâce à ce caractère, on est assuré de savoir où se termine la chaîne initiale. Nous l'ajouterons donc aux exemples, bien qu'il ne soit pas utile à la construction.

Ainsi, on peut voir l'ensemble des rotations d'une chaîne de caractères "\textsc{barbapapa\$}" dans la figure \ref{rotations}.

\begin{figure}[h!]
\fbox{%
  \begin{minipage}{\linewidth}
  	\begin{center}
 \textsc{%
      barbapapa\$\\
	  arbapapa\$b\\
	  rbapapa\$ba\\
	  bapapa\$bar\\
	  apapa\$barb\\
	  papa\$barba\\
	  apa\$barbap\\
	  pa\$barbapa\\
      a\$barbapap\\
	  \$barbapapa
	  }
	\end{center}
  \end{minipage}%
}
\caption{Ensemble des rotations de la chaîne "\textsc{barbapapa\$}"}
\label{rotations}
\end{figure}

Ordonnés par ordre lexicographique, on obtient l'exemple \ref{bwt}, où l'on ne garde que la dernière lettre pour avoir la \bwt.
\begin{figure}[h!]
\fbox{%
  \begin{minipage}{\linewidth}
    t = "\textsc{barbapapa\$}"
  	\begin{center}
  	  \textsc{%
      \$barbapap {\color{red}a}\\
      a\$barbapa {\color{red}p}\\
	  apa\$barba {\color{red}p}\\
	  apapa\$bar {\color{red}b}\\
	  arbapapa\$ {\color{red}b}\\
	  bapapa\$ba {\color{red}r}\\
	  barbapapa {\color{red}\$}\\
	  pa\$barbap {\color{red}a}\\
	  papa\$barb {\color{red}a}\\
	  rbapapa\$b {\color{red}a}
	  }
	\end{center}
	\bwt(t) = "\textsc{appbbr\$aaa}"
  \end{minipage}%
}
\caption{Construction de la \bwt\ de "\textsc{barbapapa\$}"}
\label{bwt}
\end{figure}

On peut d'ailleurs remarquer dans la \bwt\ les suites consécutives de P, B et A. En effet, lorsque les lettres sont suivies des mêmes caractères, elles vont se retrouver à côté dans la \bwt, facilitant ainsi la compression.

\subsubsection{Réversibilité}
Lorsqu'un algorithme est réversible, c'est que l'on peut retrouver sa donnée de départ à partir de son résultat. C'est par exemple le cas de la compression, qu'on inverse en décompressant. C'est donc une propriété importante de la \bwt, puisqu'elle permet de retrouver le texte de départ.


Pour inverser la \bwt, on reconstruit le tableau des rotations, colonne par colonne.
Tout d'abord, la dernière colonne correspond à la \bwt, et la première à l'ensemble des lettres dans l'ordre lexicographique. Comme ce sont des rotations, les lettres de la première colonne suivent celles de la dernière dans le texte d'origine. En triant ces groupes de deux lettres, on obtient donc la deuxième colonne, comme dans la figure \ref{unbwt}

\begin{figure}[h!]
\fbox{%
  \begin{minipage}{\linewidth}
	\bwt(t) = "\textsc{appbbr\$aaa}"
  	\begin{center}
  	  \textsc{%
      \$b . . . . . . a{\color{lightgray}\$}\\
	  a\$ . . . . . . p{\color{lightgray}a}\\
	  ap . . . . . . p{\color{lightgray}a}\\
	  ap . . . . . . b{\color{lightgray}a}\\
	  ar . . . . . . b{\color{lightgray}a}\\
	  ba . . . . . . r{\color{lightgray}b}\\
	  ba . . . . . . \${\color{lightgray}b}\\
	  pa . . . . . . a{\color{lightgray}p}\\
	  pa . . . . . . a{\color{lightgray}p}\\
	  rb . . . . . . a{\color{lightgray}r}
	  }
	\end{center}
  \end{minipage}%
}
\caption{Inversion de la \textsc{bwt}, construction de la deuxième colonne. La première colonne a été mise en gris à côté de la dernière pour voir les groupes de deux lettres que l'on retrouve triés à gauche.}
\label{unbwt} 
\end{figure}

On construit de la même façon les colonnes suivantes, pour lire la chaîne initiale sur la ligne terminant par \$.

\subsubsection{FM-index}

En 2000, Ferragina et Manzini\up{\cite{fmindex}} créent une structure d'indexation, appelée FM-index, basée sur la \bwt.

Le FM-index contient la \bwt, un échantillonnage du tableau des suffixes (qui donne la position dans le texte initial de chaque rotation), ainsi qu'une fonction $rank(c, i)$  et un tableau $C$.
La fonction $rank(c, i)$ renvoie  le nombre de caractères $c$ dans $\bwt[0..i]$. Le tableau $C$ donne pour chaque caractère de la \bwt\ le nombre de caractères lexicographiquement plus petits que lui.

Grâce à cette structure, il est possible de calculer l'association entre la \bwt\ (dernière colonne de la table des rotations, aussi désignée par L pour \textit{last}) et la première colonne de la table des rotations (désignée par F pour \textit{first}) avec la formule suivante : $LF(i) = C[i] + rank(\bwt[i], i)$ (voir figure \ref{lf}).

L'inversion de la \bwt\ peut se faire avec LF() sans la reconstruction de la totalité du tableau des rotations. 
Il suffit de partir de \$ et d'appliquer la fonction LF(), en reconstruisant la chaîne initiale jusqu'au début.

Par exemple, dans la figure \ref{lf}, '\$' est à l'indice 6, et $LF(6) = 0$. $\bwt[0] = \textsc{'a'}$, donc t se termine par \textsc{"a\$"}. On continue, $LF(0) = 1$, $t = \textsc{"...pa\$"}$ ; $LF(1) = 7$, $t = \textsc{"...apa\$"}$ ; ainsi de suite jusqu'à ce qu'on ait retrouvé l'entièreté de la chaîne de départ.

\begin{figure}[h!]
\fbox{%
  \begin{minipage}{\linewidth}
	$\bwt(t) = "\textsc{appbbr\$aaa}"$
  	\begin{center}
  	  \textsc{%
  	  \begin{tabular}{c|c|c}
  	  	\textup{i} & \bwt\ & \textsc{LF(\textup{i})}\tabularnewline \hline
        0 & \$barbapap {\color{red}a} & 1 \\
	    1 & a\$barbapa {\color{red}p} & 7 \\
	    2 & apa\$barba {\color{red}p} & 8 \\
	    3 & apapa\$bar {\color{red}b} & 5 \\
	    4 & arbapapa\$ {\color{red}b} & 6 \\
	    5 & bapapa\$ba {\color{red}r} & 9 \\
	    6 & {\color{red}barbapapa \$} & 0 \\
	    7 & pa\$barbap {\color{red}a} & 2 \\
	    8 & papa\$barb {\color{red}a} & 3 \\
	    9 & rbapapa\$b {\color{red}a} & 4
  	  \end{tabular}	
	  }
	\end{center}
	$t = "\textsc{barbapapa\$}"$
  \end{minipage}%
}
\caption{La fonction LF()}
\label{lf} 
\end{figure}

Grâce à cette structure, Ferragina et Manzini développent également un nouvel algorithme, appelé \textit{backward search}, qui émule sur la \bwt\ la recherche dans un tableau de suffixes.

Pour trouver un motif $m$, on commence par regarder la dernière lettre de $m$, qu'on appellera $a$. Les rotations commençant par cette lettre se trouvent entre les indices $s_a = C[a] + 1$ et $e_a = C[succ(a)]$, avec $succ(a)$ la fonction qui renvoie la lettre suivant $a$ dans l'alphabet. 

Puis, on prend l'avant dernière lettre de $m$, désignée par $b$. Cette fois, le préfixe $ba$ est trouvé dans les rotations $s_b = C[b] + Occ(b, s_a-1) + 1$ et $e_b = C[b] + Occ(b, e_a)$.

On répète cette opération jusqu'à ce qu'on ait lu tout le motif. Si les bornes $e_x$ et $s_x$ se croisent ou s'inversent, c'est-à-dire si $e_x - s_x \le 0$, alors $m$ n'est pas dans le texte.

Une fois ces bornes délimitées, on utilise le tableau de suffixes pour trouver les positions de $m$ dans le texte. Soit $p$ la distance d'échantillonnage du tableau de suffixe SA, la position des préfixes des rotations $r$ multiples de $p$ dans le texte est SA[$r/p$]. Pour les rotations qui ne sont pas multiples de $p$, on parcourt la fonction LF() jusqu'à trouver l'une de ces rotations. La position de $m$ se retrouve en ajoutant à la position de cette rotation le nombre d'itérations de la fonction LF() calculées pour y parvenir.
%Exemple

\subsubsection{Contexte borné}
Un inconvénient majeur de la \bwt\ est le temps et l'espace nécessaires à sa construction. En effet, la construction se base sur le tri des rotations de la chaîne. Or, la comparaison d'une seule rotation a une complexité dans le pire des cas de $n^2$, $n$ étant la longueur de la chaîne.
Pour pallier cela, Shindler en 1997 et Yoko en 1999 proposent de ne trier les suffixes que jusqu'à un rang $k$ (Figure \ref{kbwt}).

Cette nouvelle structure est couramment appelée transformée de Burrows-Wheeler à contexte borné, ou $k$-\bwt, et on appelle $k$-contextes les parties de la \kbwt\ dont les rotations sont égales jusqu'à $k$. Dans ces $k$-contextes, les rotations sont rangées dans l'ordre d'apparition dans le texte. L'information de ces $k$-contextes peut être stockée dans un vecteur de bits, que nous appellerons ici $D_k$. Elle peut aussi être retrouvée en temps voulu, comme décrit par Petri\up{\cite{petri}} dans sa thèse.

Le résultat de la \kbwt\ est très proche de la \bwt\ lorsque $k$ est grand, puisque les deux chaînes ne diffèrent que sur les $k$-contextes. Mais même pour des $k$ petits, comme $k = 10$ par exemple pour des reads de taille 70, les deux structures restent similaires. La \kbwt\ est donc également compressible facilement.

Pour inverser la \kbwt, on utilise un algorithme similaire à celui de la \bwt. La différence se situe sur les $k$-contextes, qui sont déjà dans l'ordre du texte. Ainsi, si on calcule la fonction LF() de la même façon que pour la \bwt, elle n'associe pas toujours la dernière colonne des rotations à la première, nous l'appellerons donc LF'() (Figure \ref{kbwt}). Mais l'inversion reste possible. Pour cela, on part de la fin de la chaîne, \$, et on utilise la fonction LF(). Mais au lieu de continuer avec le résultat de LF(), on prend la première rotation de son $k$-contexte sur laquelle on n'est pas encore passé.

%la fonction LF() de la même façon que dans le FM-index pour les k-contextes de taille 1. Lorsque LF() pointe sur un contexte plus grand, les $k$-contextes étant déjà dans l'ordre du texte, il suffit de prendre la première lettre, et de marquer le début du contexte à la lettre suivante.
%\rmq{Je ne comprends pas cette phrase}

\begin{figure}
\fbox{%
  \begin{minipage}{\linewidth}
  	$t = "\textsc{barbapapa\$}"$\\
	\kbwt$(t) = "\textsc{appbb{\color{red}\$r}aaa}"$\\
	$k = 2$
  	\begin{center}
  	  \textsc{%
  	  \begin{tabular}{c|c|c}
  	  	\textup{i} & \bwt\ & \textsc{LF'(\textup{i})}\tabularnewline \hline
        0 & \$b arbapap a & 1 \\
	    1 & a\$ barbapa p & 7 \\
	    2 & {\color{blue}ap} a\$barba p & 8 \\
	    3 & {\color{blue}ap} apa\$bar b & 5 \\
	    4 & ar bapapa\$ b & 6 \\
	    5 & {\color{red}ba} rbapapa {\color{red}\$} & 9 \\
	    6 & {\color{red}ba} papa\$ba {\color{red}r} & 0 \\
	    7 & {\color{orange}pa} \$barbap a & 2 \\
	    8 & {\color{orange}pa} pa\$barb a & 3 \\
	    9 & rb apapa\$b a & 4
  	  \end{tabular}	
	  }
	\end{center}
  \end{minipage}%
}
\caption{\kbwt\ de \textsc{"barbapapa\$"} pour $k = 2$. Les couleurs représentent les différents $k$-contextes. En rouge, on peut voir la différence entre la \bwt\ et la \kbwt.}
\label{kbwt}
\end{figure}



La \textit{backward search}, par contre, ne peut pas se calculer pour des motifs de longueur supérieure à $k$. Pour rechercher un motif $m$ plus grand que $k$, une méthode consiste à rechercher le suffixe de longueur $k$ de $m$, puis de remonter avec LF() le long du texte pour tester pour chaque résultat si les précédentes correspondent bien au début du motif. Pour cela par contre, la fonction LF() doit être calculée précisément. Petri\up{\cite{petri}} a trouvé un algorithme pour retrouver LF() à partir de la \kbwt, la $k-1$-\bwt, et quelques structures auxiliaires, dont des vecteurs de bits et deux colonnes supplémentaires de la matrice des rotations.
%\rmq{Précise ce que veut dire «~tester chaque résultat~»}

La \kbwt\ se calcule donc plus facilement que la \bwt, tout en se compressant autant et en assurant également sa réversibilité. Néanmoins, la recherche de motifs requiert beaucoup d'espace supplémentaire.

  \cleardoublepage
  \section{Stage}

\subsection{But}
%expliquer le sujet / les idées de départ :
%      	trier la kbwt (améliorer la compression)
%		garder les propriétés de la bwt
%		améliorer l'indexation (éviter les calculs redondants)
Le but de ce travail est de proposer un algorithme efficace de compression et d'indexation des reads peu gourmand en mémoire. Pour cela, la piste privilégiée est l'étude de la transformée de Burrows-Wheeler à contexte borné. 

Les reads étant une donnée très redondante, la recherche d'un motif dans le texte devrait occasionner de nombreux calculs identiques. L'un des objectifs est donc d'optimiser la recherche en factorisant ces calculs. 

De plus, du fait de la redondance des données, les $k$-contextes devraient être grands et nombreux. Aussi, pour accentuer la probabilité d'obtenir de longues suites de caractères identiques, l'idée est de trier la \kbwt sur ses $k$-contextes. 

Il faut donc fournir les structures et algorithmes nécessaires au maintien des propriétés de la \bwt sur la \kbwt\ triée.


\subsection{Approche}
%	implémentation de la bwt, kbwt, kbwt triée
%	étude sur l'entropie des différentes structures
%	essai de compression avec différents compresseurs existants
%	stocker LF plutot que les structures de Petri
%	optimiser le stockage de LF
%	enlever les k-contextes chevauchant 2 reads



\subsection{Résultats} 

expliquer les stuctures et algos les plus efficaces et donner les temps d'accès et taux de compression.

  \cleardoublepage
  \section*{Discussion}
\addcontentsline{toc}{section}{Conclusion}

Durant ce stage, nous avons donc étudié une façon d'améliorer la compression et l'indexation des reads grâce à la \kbwt. Pour cela, nous avons trié la \kbwt\ sur ses $k$-contextes et stocké la fonction LF(). Nous avons également construit un algorithme pour améliorer la recherche de positions pour des motifs courts. Nous avons ensuite tester nos solutions sur des jeux de données représentatifs de reads sans erreurs.

Lors de ce stage, nous avons également tenté de supprimer de la \kbwt\ les $k$-contextes contenant le symbole de fin de read \$, ceux-ci étant petits et ne représentant pas des motifs intéressants. Leur suppression aurait amélioré grandement le taux de compression, mais nous nous sommes rendus compte qu'ils étaient indispensables au fonctionnement de la structure, et cette piste n'a pas aboutie.

Pour continuer les recherches effectuées, il serait intéressant de calculer le gain de temps de construction entre la \bwt et la \kbwt\ triée.

Pour finir, ces résultats nous permettent d'envisager des améliorations intéressantes, notamment sur la recherche de motifs plus grands que $k$. L'algorithme présenté à la fin de ce rapport pourrait en effet être adapté pour vérifier l'appartenance à un motif en plus de calculer la position dans le texte. En effet, comme expliqué dans la section sur la \kbwt, pour rechercher un motif plus grand que $k$ dans la \kbwt, il faut chercher son suffixe de taille $k$ puis tester tous les résultats en remontant la fonction LF() pour vérifier leur validité. Ce parcourt de la fonction LF() pourrait donc être fait avec notre algorithme, en intégrant cette recherche la recherche des positions, pour gagner en efficacité.


  \cleardoublepage
  \phantomsection\addcontentsline{toc}{section}{Références}
\begin{thebibliography}{ABC}	
	\bibitem[1]{wt} R. Grossi, A. Gupta, and J. S. Vitter, (January 2003) \emph{High-order entropy-compressed text indexes}, Proceedings of the 14th Annual SIAM/ACM Symposium on Discrete Algorithms (SODA), 841-850.
    \bibitem[2]{bwt} Michael Burrows et D. J. Wheeler, (10th May 1994) \emph{A block-sorting lossless data compression algorithm}, Digital SRC Research Report 124.
    \bibitem[3]{petri} Matthias Petri, (July 2013) \emph{Scalable succinct indexing for large text collections}.
    \bibitem[4]{sarray} D. Okanohara et K. Sadakane, (2006) \emph{Practical entropy-compressed rank/select dictionary}, CoRR abs/cs/0610001.
    \bibitem[5]{fmindex} Paolo Ferragina et Giovanni Manzini, (2000) \emph{Opportunistic Data Structures with Applications}, Proceedings of the 41st Annual Symposium on Foundations of Computer Science. p.390.


%    \bibitem[2]{fmindex} Michael Burrows, D. J. Wheeler \emph{A block-sorting lossless data compression algorithm}, 10th May 1994, Digital SRC Research Report 124.
%    \bibitem[a]{tutoltx} tutoriel \LaTeX{} \emph{http://www.ukonline.be/programmation/latex/tutoriel/index.php}
%    \bibitem[b]{expltx} exemple de rapport de stage en \LaTeX{} \emph{http://blog.hikoweb.net/index.php?/post2011/11/06/Exemple-de-rapport-en-LaTeX}
\end{thebibliography}

\end{document}

