\section*{Résumé} % Pas de numérotation
\addcontentsline{toc}{section}{Résumé} % Ajout dans la table des matières

Le séquençage de l'ADN est depuis plus de 20 ans un sujet important de la bio-informatique, avec des applications en biologie fondamentale comme en médecine. Aussi bien pour étudier l'évolution ou des maladies, la demande est de plus en plus importante. De plus, depuis l'émergence des techniques de séquençage haut débit, la masse de données biologiques a augmenté plus vite que l'efficacité des algorithmes existants. Le besoin d'améliorer ces algorithmes est donc de plus en plus pressant, à chaque étape du traitement et de l'analyse de ces données. La compression et l'indexation des reads, fragments d'ADN retournés par les séquenceurs, est notamment un sujet pour lequel le besoin de nouvelles méthodes est important, du fait de la taille grandissante de ces données.

Beaucoup d'algorithmes sur la compression de données existent déjà, et sont utilisés pour stocker des données biologiques. Le travail présenté ici porte donc sur l'étude de l'un de ces algorithmes, \textit{la transformée de Burrows-Wheeler à contexte borné}, et son amélioration pour ce type de données particulier que sont les reads. Ce travail a été réalisé au cours du stage de fin de licence informatique à Lille 1, dans l'équipe de bio-informatique Bonsai.

%Dans le cadre de ma troisième année de licence informatique, j'ai effectué un stage de recherche à l'Inria. Du 30 mars au 30 juin, dans l'équipe Bonsai, j'ai travaillé sur l'algorithmique du texte appliquée à la bio-informatique.

%Ce stage a été pour moi l'opportunité d'approfondir mes connaissances sur la bio-informatique, ainsi que sur diverses structures de données telles que les arbres ou les tableaux de suffixes. Il m'a également permis de découvrir le monde de la recherche et de participer à des conférences.

%Mon travail a consisté essentiellement en l'étude, l'amélioration et l'implémentation d'un algorithme de compression et d'indexation de texte.


%Afin de rendre de compte de ce stage, il parait important de présenter l'équipe dans lequel il s'est déroulé, ainsi que le champ de recherche sur lequel il portait. Il apparaît ensuite nécessaire d'expliquer l'état de l'art concernant le problème posé par le sujet de stage. On pourra alors exposer le travail réaliser pendant ce stage, en revenant d'abord sur le sujet, puis en expliquant les pistes suivies, pour enfin s'intéresser aux résultats.